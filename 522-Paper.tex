\documentclass{article} % For LaTeX2e
\usepackage{nips15submit_e,times}
\usepackage{hyperref}
\usepackage{url}
%\documentstyle[nips14submit_09,times,art10]{article} % For LaTeX 2.09

\title{INCAT : Intelligence-based Cybersecurity Awareness Training}
\author{
Tam N. Nguyen, Lydia Sbityakov, Samantha Scoggins\\
North Carolina State University\\
\texttt{tam.nguyen@ncsu.edu, lesbitya@ncsu.edu, smscoggi@ncsu.edu} \\
}
\newcommand{\fix}{\marginpar{FIX}}
\newcommand{\new}{\marginpar{NEW}}
\nipsfinalcopy % Uncomment for camera-ready version
\begin{document}
\maketitle

\section{Project idea}
Cybersecurity training should be adaptable to evolving cyberthreat landscape, be cost effective and most important of all, be integrated well with other components such as enterprise risk management, incident management, threat intelligence and so on. Unfortunately, very few cybersecurity training platforms can satisfy those requirements.

This paper proposes a new model for conducting cybersecurity training with three main objectives: (i) training efforts are initiated by emerging relevant threats and delivered first to the most vulnerable groups (ii) each training session must be
promptly executed (iii) training results must be able to provide actionable intelligence to be employed by other systems such as enterprise risk management, enterprise threat intelligence, etc.

\section{Data sets and Software}
We will use a combination of three data sets which include: the National Vulnerability Database \footnote{https://nvd.nist.gov/vuln}, End-users' perceptions of cyberattack vectors (derived from 500 survey participants), and end-users' actual knowledge (derived from quiz results on the pool of 500 participants).

The following tools are being considered to process, analyze, and display the data: IBM Watson, R, SAS, and Python (using libraries:  pandas, numpy, matplotlib, sklearn).  Information from the three datasets will be layered and displayed in one dashboard.  Adaptive Qualtrics technologies will be used for surveys and quizzes.

\section{Related papers}
Literature review will include: "Social Representations of Cyber Security by University Students and Implications for Instructional Design; Integrating Concept Mapping into Information Systems Education for Meaningful Learning and Assessment" ; "A Data Analytics Approach to the Cybercrime Underground Economy" ; "Security Awareness and Effective Feedback: Categorical Behaviour vs. Reported Behaviour"

\section{Delivery schedule and division of work}

\begin{itemize}
\item Parse NVD database, establish core statistics for graphing (by Lydia and Samantha, Nov 10)
\item Relational graphs of attack vectors from NVD and User perspectives (by Tam, Nov 13th)
\item Constructing proofs of training efforts' effectiveness on addressing gaps between Users and NVD perspectives (all team members, to be completed by end of course)
\end{itemize}

\end{document}