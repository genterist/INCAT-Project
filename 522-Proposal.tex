\documentclass{article} % For LaTeX2e
\usepackage{nips15submit_e,times}
\usepackage{hyperref}
\usepackage{url}
%\documentstyle[nips14submit_09,times,art10]{article} % For LaTeX 2.09


\title{INCAT : Intelligence-based Cybersecurity Awareness Training}
\author{
Tam N. Nguyen, Lydia Sbityakov, Samantha Scoggins\\
North Carolina State University\\
\texttt{tam.nguyen, lesbitya, smscoggi@ncsu.edu} \\
}
\newcommand{\fix}{\marginpar{FIX}}
\newcommand{\new}{\marginpar{NEW}}
\nipsfinalcopy % Uncomment for camera-ready version
\begin{document}
\maketitle

\section{Project idea}
Cybersecurity training should be adaptable to evolving cyber threat landscape, be cost effective and most important of all, be integrated well with other components such as enterprise risk management, incident management, threat intelligence and so on. Unfortunately, very few cyber security training platforms can satisfy those requirements.

This paper proposes a new model for conducting cyber security trainings with three main objectives: (i) training efforts are initiated by emerging relevant threats and delivered first to respective most vulnerable members (ii) each training session must be
promptly executed (iii) training results must be able to provide actionable intelligence to be employed by other systems such as enterprise risk management, enterprise threat intelligence, etc.

\section{Data sets and Software}
We will use a combination of three data sets which include: the National Vulnerability Database \footnote{https://nvd.nist.gov/vuln}, End-users' perceptions of cyber attack vectors (derived from 500 survey participants), and End-users' actual knowledge (derived from quiz results on the pool of 500 participants).

IBM Watson, Scikit-learn, Python programming and visualizations will be used to process the data, display and layered information from three databases into one dashboard. Adaptive Qualtrics technolgies will be used for surveys and quizes.

\section{Related papers}
Literature review will include: "Social Representations of Cybersecurity by University Students and Implications for Instructional Design; Integrating Concept Mapping into Information Systems Education for Meaningful Learning and Assessment" ; "A Data Analytics Approach to the Cybercrime Underground Economy" ; "Security Awareness and Affective Feedback: Categorical Behaviour vs. Reported Behaviour"

\section{Delivery schedule and division of work}

\begin{itemize}
\item Parse NVD database, establish core statistics for graphing (by Lydia and Samantha, Nov 10th)
\item Relational graph of attack vectors from NVD perspective (by Tam, Nov 13th)
\item Relational graph of attack vectors from User perspective (by Tam, Nov 13th)
\item Constructing proofs of training efforts' effectiveness on addressing gaps between Users and NVD perspectives (all team members, to be completed by end of course)
\end{itemize}

\end{document}